\documentclass[12pt, reqno]{amsart}

\usepackage{amssymb,amsmath,amsthm, amsfonts}
\usepackage[francais]{babel}
%\usepackage[german]{babel}
%\usepackage[latin1]{inputenc}
\usepackage[all,knot]{xy}
\usepackage{graphicx}
\usepackage{tikz}
\usepackage{dsfont}

\oddsidemargin = 0.5cm 
\evensidemargin = 0.5cm 
\textwidth = 16cm
\headsep = -1cm
\textheight = 22cm

\newtheorem{case}{Case}
\newcommand{\caseif}{\textnormal{if }}
\newcommand{\leg}[2]{\genfrac{(}{)}{}{}{#1}{#2}}
\newcommand{\bfrac}[2]{\genfrac{}{}{}{0}{#1}{#2}}
\newcommand{\sm}[4]{\left(\begin{smallmatrix}#1&#2\\ #3&#4 \end{smallmatrix} \right)}
\newtheorem{theorem}{Theorem}
\newtheorem{lemma}[theorem]{Lemma}
\newtheorem{corollary}[theorem]{Corollary}
\newtheorem*{conjecture}{\bf Conjecture}
\newtheorem{proposition}[theorem]{Proposition}
\newtheorem{definition}[theorem]{Definition}
%\renewcommand{\theequation}{\thesection.\arabic{equation}}
\renewcommand{\thetheorem}{\thesection.\arabic{theorem}}
\theoremstyle{definition}
\newtheorem{exercise}{Exercise}
\newtheorem{bonus_exercise}[exercise]{Exercise*}
\newtheorem*{solution}{Solution}
\newtheorem*{answer}{Answer}
\theoremstyle{remark}
\newtheorem*{claim}{Claim}
%\newtheorem*{proof}{Proof}
\newtheorem*{theoremno}{{\bf Theorem}}
\newtheorem*{remark}{Remark}
\newtheorem*{hint}{Hint}
\newtheorem*{example}{Example}
\numberwithin{theorem}{section}
% \numberwithin{equation}{section}
\newtheorem*{theorem*}{Theorem}

\newcommand{\C}{\mathbb{C}}
\newcommand{\R}{\mathbb{R}}
\newcommand{\N}{\mathbb{N}}
\newcommand{\Q}{\mathbb{Q}}
\newcommand{\Z}{\mathbb{Z}}
\newcommand{\equirel}[2]{[#1,#2]\in\Omega}
\def\Box{\operatorname{Box}}
\newcommand{\X}{\mathcal{X}}
\newcommand{\Nc}{\mathcal{N}}
\newcommand{\cond}[1]{\mathds{1}_{\{#1\}}}


\begin{document}

\noindent
Name: Alexandre Cordonnier \\
SCIPER: 310692 \\
\begin{center}
Exercice à rendre 1 - Week 3 \\
Statistique MATH-240 \\ Spring 2021
\end{center}

\begin{exercise}
	Supposons que la fonction $f_{0}(x)$ est la densité d'une variable aléatoire $X \in \mathcal{X}$ (par exemple, la distribution uniforme sur l'intervalle $[0,1]$), et soit $s(X): \mathcal{X} \rightarrow \mathbb{R}^{d}$ une fonction de $X$. Définissons une famille de distributions de probabilité comme suit: \begin{equation}f(x ; \theta)=\frac{e^{s(x)^{T} \theta} f_{0}(x)}{\int e^{s(x)^{T} \theta} f_{0}(x) d x}\end{equation} pour toute valeur de $\theta \in \mathcal{N} \subset \mathbb{R}^{d}$, o\`u $\mathcal{N}=\left\{\theta: \ln \int e^{s(x)^{T} \theta} f_{0}(x) d x<\infty\right\}$\\
	(\emph{exponential tilting}).
	\begin{enumerate}
		\item[(a)] 
		Montrer que la transformation (1) définit une famille exponentielle.
		\item[(b)] 
		Supposons que $f_{0}(x)=1$ si $x \in[0,1],$ et 0 sinon (distribution uniforme). Supposons aussi que $s(x)=x$. Trouver l'ensemble $\mathcal{N}$ des paramètres $\theta$ possibles. Trouver la famille exponentielle générée (une telle famille exponentielle, engendrée en utilisant $s(x)=x,$ est appelée famille exponentielle naturelle). Dessiner le graphique de la densité générée pour $\theta=-1,0$ et 3.
		\item[(c)] 
		Supposons à nouveau que $f_{0}(x)=1$ si $x \in[0,1],$ et 0 sinon (distribution uniforme), et supposons maintenant $s(x)=\{\log x, \log (1-x)\}^{T}$ qui engendre une famille exponentielle à deux paramètres. Trouver l'ensemble $\mathcal{N}$ des paramètres $\theta$ possibles. Trouver la famille exponentielle générée (distribution beta).
	\end{enumerate}

\end{exercise}

\begin{solution}
	\begin{enumerate}
		\item[(a)]
		On a, $\forall x\in\X, \theta\in\Nc\subset \R^d$,
		\begin{align*}
			f(x;\theta) &= \frac{e^{s(x)^{T} \theta} f_{0}(x)}{\int e^{s(t)^{T} \theta} f_{0}(t) dt}\\
						&= \exp\left\{s(x)^T\theta + \ln(f_0(x)) - \ln\int e^{s(t)^{T}\theta}f_{0}(t)dt\right\}\\
						&= \exp\left\{\langle T(x)\,,\theta\rangle - \gamma(\theta) + S(x)\right\}
		\end{align*}
		où on a définit : \begin{align*}
			T : \X\rightarrow\R^d, &\quad T(x) = s(x)\\
			\gamma : \R^d \rightarrow \R, &\quad \gamma(\theta)=\ln\int e^{s(t)^{T}\theta}f_{0}(t)dt\\
			S : \R^d\rightarrow\R, &\quad S(x) = \ln(f_0(x))
		\end{align*}
		On remarque que le support de $f$ (l’ensemble sur lequel elle est positive) ne dépend pas de $\theta$, puisque les fonctions $\exp$ et $f_0$ sont toutes les deux positives indépendamment de $\theta$.\\
		La transformation (1) définit donc bien une famille exponentielle.
		\item[(b)]
		On a $\Nc=\left\{\theta\in\R: \ln \int e^{s(x)^{T} \theta} f_{0}(x) d x<\infty\right\}$. Comme $f_0(x) = \cond{x\in[0,1]}$ et $s(x)=x$, on a :
		\begin{align*}
			\int_\R e^{s(t)\theta}f_0(t)dt &= \int_\R e^{\theta t}\cond{x\in[0,1]}dt\\
			&= \int_0^1e^{\theta t}dt\\
			&= \left\{
				\begin{aligned}
					1 \quad&\text{si  } \theta=0\\
					\frac1\theta(e^\theta-1) \quad&\text{sinon}
				\end{aligned}
				\right.
		\end{align*}
		qui est continue en $\theta=0$, donc $\Nc=\R$ tout entier puisque $\int_0^1e^{\theta t}dt > 0$ pour tout $\theta\in\R$.
		On peut maintenant écrire \begin{align*}
			f(x;\theta) &= \frac{e^{s(x)\theta}f_0(x)}{\int e^{s(t)\theta}f_0(t)dt}\\
			&= \frac{e^{\theta x}\cdot\cond{x\in[0,1]}}{\int_0^1 e^{\theta t}dt}\\
			&= \left\{
				\begin{aligned}
					1 \cdot\cond{x\in[0,1]} \quad&\text{si  } \theta=0\\
					\frac{\theta e^{\theta x}}{e^\theta-1}\cdot\cond{x\in[0,1]} \quad&\text{sinon}
				\end{aligned}
				\right.
		\end{align*}

		On peut maintenant plot $f(x;\theta)$ pour $\theta = -1, 0, 3$:
		\begin{itemize}
			\item{$\theta = -1$:}
			\begin{figure}[h!]
				\includegraphics[width=0.90\textwidth]{plot1.PNG}
			\end{figure}
			\item{$\theta = 0$:}
			\begin{figure}[h!]
				\includegraphics[width=0.90\textwidth]{plot2.PNG}
			\end{figure}
			\item{$\theta = 3$:}
			\begin{figure}[h!]
				\includegraphics[width=0.90\textwidth]{plot3.PNG}
			\end{figure}
		\end{itemize}
		\item[(c)]
		On a $\Nc=\left\{(\theta : \ln \int e^{s(x)^{T} \theta} f_{0}(x) d x<\infty\right\}$. Comme $f_0(x) = \cond{x\in[0,1]}$ et $s(x)= (\ln(x), ln(1-x))^T$, on a pour $\theta = (\theta_1, \theta_2)^T\in\R^2$:
		\begin{align*}
			\int_\R e^{s(t)^T\theta}f_0(t)dt &= \int_\R e^{\theta_1\cdot\ln(t) + \theta_2\cdot\ln(1-t)}\cond{x\in[0,1]}dt\\
			&= \int_0^1t^{\theta_1}(1-t)^{\theta_2}dt\\
			&= \mathrm{B}(\theta_1+1, \theta_2+1)
		\end{align*}
		par définition de la fonction Bêta B$(x,y) = \int_0^1t^{x-1}(1-t)^{y-1}dt$, qui converge pour $x,y>0$. Donc $\int_\R e^{s(t)^T\theta}f_0(t)dt$ converge pour $\theta_1,\theta_2>-1$ et est strictement positif, donc $\Nc=\left\{(x,y)^T\in\R^2 : x,y>-1\right\}$.
		Maintenant, on a pour $\theta = (\theta_1, \theta_2)^T\in\Nc$ :
		\begin{align*}
			f(x;\theta) &= \frac{e^{s(x)^{T} \theta} f_{0}(x)}{\int e^{s(t)^{T} \theta} f_{0}(t) dt}\\
						&= \frac{e^{\theta_1\cdot\ln(x) + \theta_2\cdot\ln(1-x)}\cdot\cond{x\in[0,1]}}{\mathrm{B}(\theta_1+1,\theta_2+1)}\\
						&= \frac{\Gamma(\theta_1+\theta_2+2)}{\Gamma(\theta_1+1)\Gamma(\theta_2+1)}\cdot x^{\theta_1}(1-x)^{\theta_2}\cdot\cond{x\in[0,1]}
		\end{align*}
		en utilisant que B$(x,y)=\frac{\Gamma(x+y)}{\Gamma(x)\Gamma(y)}$ pour $x,y>-1$.
		
		Pour $\theta_1,\theta_2\in\N$, on peut utiliser que $\Gamma(n+1)=n!$ pour arriver à:
		\begin{align*}
			f(x;(\theta_1,\theta_2)^T) &= \frac{\Gamma(\theta_1+\theta_2+2)}{\Gamma(\theta_1+1)\Gamma(\theta_2+1)}\cdot x^{\theta_1}(1-x)^{\theta_2}\cdot\cond{x\in[0,1]}\\
			&= (\theta_1+\theta_2+1)\frac{(\theta_1+\theta_2)!}{\theta_1!\cdot\theta_2!}\cdot x^{\theta_1}(1-x)^{\theta_2}\cdot\cond{x\in[0,1]}\\
			&= (\theta_1+\theta_2+1)\binom{\theta_1+\theta_2}{\theta_1}\cdot x^{\theta_1}(1-x)^{\theta_2}\cdot\cond{x\in[0,1]}
		\end{align*}
	\end{enumerate}
\end{solution}

\end{document}