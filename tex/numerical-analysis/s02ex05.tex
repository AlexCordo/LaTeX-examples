\documentclass[12pt]{amsart}

\usepackage{amssymb,amsmath,amsthm, amsfonts}
%\usepackage[german]{babel}
\usepackage[latin1]{inputenc}
\usepackage[all,knot]{xy}
\usepackage{graphicx}
\usepackage{tikz}

\usepackage[numbered,framed]{matlab-prettifier}

\oddsidemargin = 0.5cm 
\evensidemargin = 0.5cm 
\textwidth = 16cm
\headsep = -1cm
\textheight = 22.3cm

\newtheorem{case}{Case}
\newcommand{\caseif}{\textnormal{if }}
\newcommand{\leg}[2]{\genfrac{(}{)}{}{}{#1}{#2}}
\newcommand{\bfrac}[2]{\genfrac{}{}{}{0}{#1}{#2}}
\newcommand{\sm}[4]{\left(\begin{smallmatrix}#1&#2\\ #3&#4 \end{smallmatrix} \right)}
\newtheorem{theorem}{Theorem}
\newtheorem{lemma}[theorem]{Lemma}
\newtheorem{corollary}[theorem]{Corollary}
\newtheorem*{conjecture}{\bf Conjecture}
\newtheorem{proposition}[theorem]{Proposition}
\newtheorem{definition}[theorem]{Definition}
%\renewcommand{\theequation}{\thesection.\arabic{equation}}
\renewcommand{\thetheorem}{\thesection.\arabic{theorem}}
\theoremstyle{definition}
\newtheorem{exercise}{Exercise}
\newtheorem{problem}{Problem}
\newtheorem{bonus_exercise}[exercise]{Exercise*}
\newtheorem*{solution}{Solution}
\newtheorem*{answer}{Answer}
\newtheorem*{claim}{Claim}
\theoremstyle{remark}
\newtheorem*{theoremno}{{\bf Theorem}}
\newtheorem*{remark}{Remark}
\newtheorem*{hint}{Hint}
\newtheorem*{example}{Example}
\numberwithin{theorem}{section}
% \numberwithin{equation}{section}
\newtheorem*{theorem*}{Theorem}

\newcommand{\C}{\mathbb{C}}
\newcommand{\R}{\mathbb{R}}
\newcommand{\N}{\mathbb{N}}
\newcommand{\Q}{\mathbb{Q}}
\newcommand{\Z}{\mathbb{Z}}

\begin{document}

\noindent
Name: Alexandre Cordonnier \\
SCIPER: 310692 \\
\begin{center}
Exercises - Week 2 \\
Numerical analysis MATH-250 \\ Spring 2021
\end{center}
\stepcounter{problem}
\stepcounter{problem}
\stepcounter{problem}
\stepcounter{problem}
\begin{problem}
	\underline{Section 1:} Theory


	\emph{We assume here $f$ has non-zero second derivative.}
	\begin{enumerate}
		\item[(a)]
		As $f$ is $C^2$ in a neighborhood of $x\in\R$, we can use Taylor theorem to get, for $h>0$, $f(x+h)=f(x)+f'(x)h+\frac12f''(\theta_h)h^2$ for some $\theta_h\in(x,x+h)$, which implies $\frac{f(x+h)-f(x)}{h}-f'(x)=\frac12f''(\theta_h)h$, so we have $$\left|\frac{f(x+h)-f(x)}{h}-f'(x)\right|\leq Ch$$ where $C=\frac12\left|\lim\limits_{h\to 0}f''(\theta_h)\right|=\frac12\left|f''(x)\right|$ as $f''$ is continuous at $x$ by assumption.
		\item[(b)]
		For $|\delta_1|,|\delta_2|\leq u$, we have
		\begin{align*}
			&\left|\frac{f(x+h)(1+\delta_1)-f(x)(1+\delta_2)}{h}-\frac{f(x+h)-f(x)}{h}\right|\\
			 &= \frac1h\left|f(x+h)(1+\delta_1)-f(x)(1+\delta_2)-f(x+h)-f(x)\right|\\
			&=\frac1h\left|f(x+h)\delta_1-f(x)\delta_2\right|\\
			&\leq\frac1h\left(\left|f(x+h)\right||\delta_1|+\left|f(x)\right||\delta_2|\right)\\
			&\leq\frac u{h}\left(|f(x+h)|+|f(x)|\right)
		\end{align*}
		Using continuity of $f$ at $x$, we can define $c=\lim\limits_{h\to0}\left(|f(x+h)|+|f(x)|\right)=2|f(x)|$ so we have $$\left|\frac{f(x+h)(1+\delta_1)-f(x)(1+\delta_2)}{h}-\frac{f(x+h)-f(x)}{h}\right|\leq c\frac{u}{h}$$
		\item[(c)]
		Using upper bounds found in (a) and (b), we have
		\begin{align*}
			&\left|\frac{f(x+h)(1+\delta_1)-f(x)(1+\delta_2)}{h}-f'(x)\right|\\
			&=\left|\frac{f(x+h)(1+\delta_1)-f(x)(1+\delta_2)}{h}-\frac{f(x+h)-f(x)}{h}+\frac{f(x+h)-f(x)}{h}-f'(x)\right|\\
			&\leq\left|\frac{f(x+h)(1+\delta_1)-f(x)(1+\delta_2)}{h}-\frac{f(x+h)-f(x)}{h}\right|+\left|\frac{f(x+h)-f(x)}{h}-f'(x)\right|\\
			&\leq Ch + c\frac{u}{h}
		\end{align*}
		\item[(d)]
		We want to minimize $g(h) = Ch+c\frac{u}{h}$: we have $g'(h) = C - c\frac{u}{h^2}$, with extrema at $h_{min}=\sqrt{\frac{cu}{C}}$. As $g''(h) = 2\frac{cu}{h^3} >0$ for $h>0$, $h_{min}$ is in fact the minimum. Using the expressions of $c$ and $C$ we found earlier, we have $\displaystyle h_{min}=2\cdot\sqrt{\frac{u|f(x)|}{|f''(x)|}}$.\\
		In order to verify the optimality of the theorical $h_{min}$ on an example, say $f(x)=\exp(x)$, we can plot the absolute error between the computed finite difference quotient and $\exp$ at $x=1$. In this case, theorical $h_{min}$ simplifies to $2\cdot\sqrt{u}$, with $u$ being the unit roundoff, defined as $u=\beta^{t-1}/2=\epsilon/2$. So $h_{min}=\sqrt{2\epsilon}$.\\
		\lstinputlisting[style=Matlab-editor]{error_h.m}
		Running file above (\verb|error_h.m|) yields the following plot,\\ and outputs \verb|Theorical minimum: h_min = 2.107342e-08|\begin{figure}[h!]
			\includegraphics[width=\linewidth]{plot.jpg}
		\end{figure}

		The plot clearly shows a minimum at around \verb|2.1e^-08|, which is satisfying considering we computed $h_{min}$ to be \verb|2.107342e-08|.
	\end{enumerate}

	\underline{Section 2:} MATLAB

	\begin{enumerate}
		\item[(a)]
		Here is the code of function \verb|roots2(p,q)|, returning roots of quadratic $x^2+p\cdot x+q$ in an array $[x_+,x_-]$. File \verb|roots2.m|:
		\lstinputlisting[style=Matlab-editor]{roots2.m}
		\item[(a),(b),(c)]
		Running the following code (file \verb|main.m|) produces a table displaying the exact and computed roots, together with the relative error and the evaluation of the roots, for each of the three quadratics \verb|P1|,\verb|P2| and \verb|P3|. See outputs next page.
		\lstinputlisting[style=Matlab-editor]{main.m}
		\newpage
		Table displayed:

		
		\begin{verbatim}
               Exact        Computed      Relative error    Eval of computed
            ___________    ___________    ______________    ________________

P1 root+           -0.5           -0.5          0                  0        
P1 root-             -2             -2          0                  0        
P2 root+          1e+10          1e+10          0                  1        
P2 root-          1e-10              0          1                  1        
P3 root+    -9.3132e-10              0          1                  1        
P3 root-    -1.0737e+09    -1.0737e+09          0                  1        
		\end{verbatim}


		\item[(d)]
		For quadratic \verb|P2| and \verb|P3|, we have in magnitude \verb|m|$\gg$\verb|q| (where \verb|m=-p/2|), so when function \verb|roots2| computes \verb|m^2-q|, \verb|q| may be so small compared to \verb|m^2| that the operation \verb|m^2-q| returns \verb|m^2| because of the roundoff. Even if \verb|sqrt(m^2-q)| is actually computed somewhat accurately, the difference \verb|abs(m)-sqrt(m^2-q)| is really small (since $|x|-\sqrt{x^2-y}<|y/x|$, so for $x\gg y$ the result is very small), so it may be rouded off to 0.

		As we said before, the root of the inaccuracy is when computing the value of $\pm$\verb|(abs(m)-sqrt(m^2-q))|, which happens for $x_+$ when \verb|m| is negative, i.e. \verb|p| is positive, and for $x_-$ when \verb|m| is positive, i.e. \verb|p| is negative. So the well approximated computed root is the one of the opposite sign of \verb|p|, which is verified by the above examples.
		\item[(e)] 
		Using \verb|x_bad=q/x_good|, we recover \verb|x_bad| for \verb|P2| and \verb|P3|, and evaluate \verb|P2(x2_bad)| and \verb|P3(x3_bad)| as follows:
		\lstinputlisting[style=Matlab-editor]{e.m}
		The output is

		\begin{verbatim}
			Evaluation of recovered root for quad P2: 0
			Evaluation of recovered root for quad P3: 0
		\end{verbatim}
		We observe that the method to recover the poorly-approximated roots from the well-approximated ones works well, as we have here \verb|P2(x2_bad)=0| and \verb|P3(x3_bad)=0|.
	\end{enumerate}
\end{problem}

\end{document}