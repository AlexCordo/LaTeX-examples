\documentclass[12pt]{amsart}

\usepackage{amssymb,amsmath,amsthm, amsfonts}
%\usepackage[german]{babel}
\usepackage[latin1]{inputenc}
\usepackage[all,knot]{xy}
\usepackage{graphicx}
\usepackage{tikz}

\oddsidemargin = 0.5cm 
\evensidemargin = 0.5cm 
\textwidth = 16cm
\headsep = -1cm
\textheight = 22cm

\newtheorem{case}{Case}
\newcommand{\caseif}{\textnormal{if }}
\newcommand{\leg}[2]{\genfrac{(}{)}{}{}{#1}{#2}}
\newcommand{\bfrac}[2]{\genfrac{}{}{}{0}{#1}{#2}}
\newcommand{\sm}[4]{\left(\begin{smallmatrix}#1&#2\\ #3&#4 \end{smallmatrix} \right)}
\newtheorem{theorem}{Theorem}
\newtheorem{lemma}[theorem]{Lemma}
\newtheorem{corollary}[theorem]{Corollary}
\newtheorem*{conjecture}{\bf Conjecture}
\newtheorem{proposition}[theorem]{Proposition}
\newtheorem{definition}[theorem]{Definition}
%\renewcommand{\theequation}{\thesection.\arabic{equation}}
\renewcommand{\thetheorem}{\thesection.\arabic{theorem}}
\theoremstyle{definition}
\newtheorem{exercise}{Exercise}
\newtheorem{bonus_exercise}[exercise]{Exercise*}
\newtheorem*{solution}{Solution}
\newtheorem*{answer}{Answer}
\theoremstyle{remark}
\newtheorem*{claim}{Claim}
%\newtheorem*{proof}{Proof}
\newtheorem*{theoremno}{{\bf Theorem}}
\newtheorem*{remark}{Remark}
\newtheorem*{hint}{Hint}
\newtheorem*{example}{Example}
\numberwithin{theorem}{section}
% \numberwithin{equation}{section}
\newtheorem*{theorem*}{Theorem}

\newcommand{\C}{\mathbb{C}}
\newcommand{\R}{\mathbb{R}}
\newcommand{\N}{\mathbb{N}}
\newcommand{\Q}{\mathbb{Q}}
\newcommand{\Z}{\mathbb{Z}}
\def\Box{\operatorname{Box}}
\def\Vol{\operatorname{Vol}}
\newcommand{\m}{\mathrm{m}}
\def\c{^{\mathsf{c}}}
\newcommand{\eps}{\varepsilon}


\begin{document}

\noindent
Name: Alexandre Cordonnier \\
SCIPER: 310692 \\
\begin{center}
Exercises - Week 3 \\
Analysis IV MATH-205 \\ Spring 2021
\end{center}

\stepcounter{exercise}

\begin{exercise}
	Let $f:\R\rightarrow\R$ be increasing or decreasing, then $f$ is measurable.
\end{exercise}

\begin{solution}
	Define $G_\alpha = \left\{x\in\R|f(x)>\alpha\right\}$ for all $\alpha\in\R$. We notice that $f^{-1}(]\alpha,+\infty[) = \left\{x\in\R|f(x)>\alpha\right\} = G_\alpha$, so in order to prove that $f$ is measurable, it suffices to show that $G_\alpha$ is measurable $\forall\alpha\in\R$.

	If $G_\alpha=\emptyset$, we're done as $\emptyset$ is measurable.

	If $G_\alpha\neq\emptyset$, we distinguish between the two cases increasing/decreasing, although the reasoning is quite similar (we could also have shown and used the fact that if $f$ is measurable, then $-f$ also is):
	\begin{itemize}
		\item{Case $f$ increasing:}		
		We have that $\forall x\in G_\alpha, y>x\Rightarrow f(y)\geq f(x)>\alpha \Rightarrow y\in G_\alpha$, so $[x,+\infty[\subseteq G_\alpha$ for all $x\in G_\alpha$. So we either have $G_\alpha=]s,+\infty[$ or $G_\alpha=[s,+\infty[$, where we denoted $\inf G_\alpha$ by $s$.

		\item{Case $f$ decreasing:}
		We have that $\forall x\in G_\alpha, y<x\Rightarrow f(y)\geq f(x)\geq\alpha \Rightarrow y\in G_\alpha$, so $]-\infty,x]\subseteq G_\alpha$ for all $x\in G_\alpha$. So we either have $G_\alpha=]-\infty,m[$ or $G_\alpha=]-\infty,m]$, where we denoted $\sup G_\alpha$ by $m$.
	\end{itemize}
	Either way, $G_\alpha$ is measurable as open\&closed sets are measurable, so we can conclude that $f$ is measurable.
\end{solution}



\stepcounter{exercise}
\stepcounter{exercise}
\begin{exercise}
	We want to compute the measure of a intersection of a countable family of decreasing sets.
	\begin{enumerate}
		\item[(i)]
		Show that if $A_{1} \supseteq A_{2} \supseteq A_{3} \ldots$ is a decreasing sequence of measurable sets (that is $A_{j} \supseteq A_{j+1}$ for every $j \geq 1)$ and $\m\left(A_{1}\right)<+\infty,$ then $$\m\left(\bigcap_{j=1}^{\infty} A_{j}\right)=\lim _{j \rightarrow \infty} \m\left(A_{j}\right)$$
		\item[(ii)]
		Show that the previous equality fails without the assumption $\mathrm{m}\left(A_{1}\right)<+\infty$.
	\end{enumerate}
\end{exercise}
\begin{solution}
	\begin{enumerate}
		\item[(i)]
		We first prove this useful claim:
		\begin{claim}
			If $A_{1} \supseteq A_{2} \supseteq A_{3} \ldots$ is a decreasing sequence of sets, then we have the equality (where $\sqcup$ symbolizes a union of disjoint sets) : $$A_1 = \left(\bigcap_{j\in\N}A_j\right)\sqcup\left(\bigsqcup_{j\in\N}(A_j\backslash A_{j+1})\right)$$ 
		\end{claim}
		\begin{proof}
			We first prove that the unions that are said to be disjoint are indeed disjoint:
			\begin{itemize}
				\item{$\bigcup_{j\in\N}(A_j\backslash A_{j+1})$ is a disjoint union:} For $i>j\in\N$, we have $\left(A_j\backslash A_{j+1}\right)\cap\left(A_i\backslash A_{i+1}\right)=A_j\cap A_{j+1}\c\cap A_i\cap A_{i+1}\c=\left(A_i\backslash A_{j+1}\right)\cap\left(A_j\backslash A_{i+1}\right)=\emptyset$ as $i>j\implies i\geq j+1\implies A_i\subseteq A_{j+1}\implies A_i\backslash A_{j+1}=\emptyset$.
				Since the terms are pairwise disjoint, the union is indeed a disjoint union.
				\item{$\left(\bigcap_{j\in\N}A_j\right)\cup\left(\bigsqcup_{j\in\N}(A_j\backslash A_{j+1})\right)$ is a disjoint union:} Suppose $\exists x\in\left(\bigcap_{j\in\N}A_j\right)\sqcup\left(\bigsqcup_{j\in\N}(A_j\backslash A_{j+1})\right)$. Then $x\in A_j\forall j\in\N$ and $\exists i\in\N$ such that $x\in A_i\backslash A_{i+1}$ which means $x\notin A_{i+1}$, so such an $x$ does not exist. Hence the intersection is empty, i.e. the two sets are disjoint.
			\end{itemize}
			\noindent
			Now, we can prove the equality by double inclusion:
			\begin{itemize}
				\item{"$\supseteq$":} $x\in\left(\bigcap_{j\in\N}A_j\right)\implies x\in A_1$ as $A_{1} \supseteq A_{2} \supseteq \ldots$\\
				$x\in\left(\bigsqcup_{j\in\N}(A_j\backslash A_{j+1})\right)\implies\exists j\in\N$ such that $x\in A_j\backslash A_{j+1}\subseteq A_j\subseteq A_1$.\\
				So we conclude that $A_1 \supseteq \left(\bigcap_{j\in\N}A_j\right)\sqcup\left(\bigsqcup_{j\in\N}(A_j\backslash A_{j+1})\right)$.
				\item{"$\subseteq$":} For $x\in A_1$, we define $s_x=\sup\left\{n\in\N | x\in A_n\right\}$. Now,\\
				$s_x=\infty \implies x\in A_j\forall j\in\N$, i.e. $x\in\left(\bigcap_{j\in\N}A_j\right)$.\\
				$s_x=n$ for some $n\in\N$ $\implies x\in A_n$ and $x\notin A_{n+1}\implies x\in (A_n\backslash A_{n+1})\subset \bigsqcup_{j\in\N}(A_j\backslash A_{j+1})$.\\
				So we conclude that $A_1 \subseteq \left(\bigcap_{j\in\N}A_j\right)\sqcup\left(\bigsqcup_{j\in\N}(A_j\backslash A_{j+1})\right)$.
			\end{itemize}
		\end{proof}
		We can use the identity of the above claim, and apply $\m$ to it as sets $\{A_j\}_{j\in\N}$ are measurable. We can write
		\begin{align*}
			\m\left(A_1\right) &= \m\left(\left(\bigcap_{j\in\N}A_j\right)\sqcup\left(\bigsqcup_{j\in\N}(A_j\backslash A_{j+1})\right)\right)\\
			&=\m\left(\bigcap_{j\in\N}A_j\right)+\m\left(\bigsqcup_{j\in\N}(A_j\backslash A_{j+1})\right)
		\end{align*} by finite additivity of $\m$. As by assumption, $\m(A_1)$ is finite, we can move it around:
		\begin{align*}
			\m\left(\bigcap_{j\in\N}A_j\right) &= \m(A_1) - \m\left(\bigsqcup_{j\in\N}(A_j\backslash A_{j+1})\right)\\
			&= \m(A_1) - \sum_{j\in\N}\m(A_j\backslash A_{j+1}) \quad\quad\text{ by countable additivity of m}\\
			&= \m(A_1) - \lim_{n\to\infty}\sum_{j=1}^n\m(A_j\backslash A_{j+1})\\
			&= \m(A_1) - \lim_{n\to\infty}\underbrace{\sum_{j=1}^n\left[\m(A_j)-m(A_{j+1})\right]}_{=\m(A_1)-\m(A_{n+1})} \quad\quad\text{as }A_{j+1}\subseteq A_j\\
			&= \lim_{n\to\infty} \m(A_{n+1}) \quad\text{as the above sum telescopes, leaving only $\lim_{n\to\infty}\left(\m(A_1)-\m(A_{n+1})\right)$}
		\end{align*}
		We can now conclude that $\m\left(\bigcap\limits_{j=1}^{\infty} A_{j}\right)=\lim\limits_{j \to \infty} \m\left(A_{j}\right)$.
		\item[(ii)]
		We can take the typical "escaping half-line" counter example: define $A_j=]j,+\infty[$. For all $j\in\N$, we have $A_j\supseteq A_{j+1}$, and $A_j$ is measurable as it is open in $\R$. We then have $\m(A_j)=\m(]j,+\infty[) = +\infty$, so $\lim\limits_{j \to \infty} \m\left(A_{j}\right)=+\infty$. \\
		But as we have $\bigcap\limits_{j=1}^{\infty} A_{j} = \emptyset$, the equality fails:
		\[
			0 = \m\left(\bigcap\limits_{j=1}^{\infty} A_{j}\right) \neq \lim\limits_{j \to \infty} \m\left(A_{j}\right)=+\infty
		\]
	\end{enumerate}
\end{solution}
\end{document}