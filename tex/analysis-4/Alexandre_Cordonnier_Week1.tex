\documentclass[12pt]{amsart}

\usepackage{amssymb,amsmath,amsthm, amsfonts}
%\usepackage[german]{babel}
\usepackage[latin1]{inputenc}
\usepackage[all,knot]{xy}
\usepackage{graphicx}
\usepackage{tikz}

\oddsidemargin = 0.5cm 
\evensidemargin = 0.5cm 
\textwidth = 16cm
\headsep = -1cm
\textheight = 22cm

\newtheorem{case}{Case}
\newcommand{\caseif}{\textnormal{if }}
\newcommand{\leg}[2]{\genfrac{(}{)}{}{}{#1}{#2}}
\newcommand{\bfrac}[2]{\genfrac{}{}{}{0}{#1}{#2}}
\newcommand{\sm}[4]{\left(\begin{smallmatrix}#1&#2\\ #3&#4 \end{smallmatrix} \right)}
\newtheorem{theorem}{Theorem}
\newtheorem{lemma}[theorem]{Lemma}
\newtheorem{corollary}[theorem]{Corollary}
\newtheorem*{conjecture}{\bf Conjecture}
\newtheorem{proposition}[theorem]{Proposition}
\newtheorem{definition}[theorem]{Definition}
%\renewcommand{\theequation}{\thesection.\arabic{equation}}
\renewcommand{\thetheorem}{\thesection.\arabic{theorem}}
\theoremstyle{definition}
\newtheorem{exercise}{Exercise}
\newtheorem{bonus_exercise}[exercise]{Exercise*}
\newtheorem*{solution}{Solution}
\newtheorem*{answer}{Answer}
\theoremstyle{remark}
\newtheorem*{claim}{Claim}
%\newtheorem*{proof}{Proof}
\newtheorem*{theoremno}{{\bf Theorem}}
\newtheorem*{remark}{Remark}
\newtheorem*{hint}{Hint}
\newtheorem*{example}{Example}
\numberwithin{theorem}{section}
% \numberwithin{equation}{section}
\newtheorem*{theorem*}{Theorem}

\newcommand{\C}{\mathbb{C}}
\newcommand{\R}{\mathbb{R}}
\newcommand{\N}{\mathbb{N}}
\newcommand{\Q}{\mathbb{Q}}
\newcommand{\Z}{\mathbb{Z}}
\newcommand{\equirel}[2]{[#1,#2]\in\Omega}
\def\Box{\operatorname{Box}}


\begin{document}

\noindent
Name: Alexandre Cordonnier \\
SCIPER: 310692 \\
\begin{center}
Exercises - Week 1 \\
Analysis IV MATH-205 \\ Spring 2021
\end{center}

\stepcounter{exercise}
\stepcounter{exercise}

\begin{exercise}
	We show that open sets in $\mathbb{R}^{n}$ are countable unions of open boxes which can be asked to be disjoint if $n=1$.
	\begin{enumerate}
		\item[(i)] Show that if $$\left\{I_{\alpha}\right\}_{\alpha \in A}$$ is a family of open intervals, then there is a countable subfamily $\left\{I_{k}\right\}_{k=1}^{\infty}$ such that $$ \bigcup_{k=1}^{\infty} I_{k}=\bigcup_{\alpha \in A} I_{\alpha} $$
		\item[(ii)] Let $\Omega \subset \mathbb{R}$ be an open set. Show that $\Omega$ can be written as a countable union of open disjoint intervals.
		\item[(iii)] (Generalization of (ii) to n dimensions) Prove that every open set $\Omega \subset \mathbb{R}^{n}$ can be written as a countable union of open boxes.
	\end{enumerate}
\end{exercise}

\begin{solution}
	\begin{enumerate}
		\item[(i)]
		Let $x\in\Omega := \bigcup_{\alpha\in A}I_{\alpha}$. We have $x\in I_{\alpha}$ for some $\alpha\in A$. As $I_{\alpha}$ is open, we can find  an open interval containing $x$ and contained in $I_{\alpha}$. As $\Q$ is dense in $\R$ (which in particular means that $a<b\in\R \Rightarrow \ \exists q\in\Q$ such that $a<q<b$), we can choose the interval's extrema to belong to $\Q$. So for all $x\in\Omega$ we can find an interval $I_x$ such that $$x\in I_x\subseteq I_{\alpha} \text{ and } I_x = ]a_x, b_x[ \text{, with } a_x,b_x\in\Q$$
		We can easily see that $\Omega = \bigcup_{x\in \Omega}I_{x}$, by construction of $I_x$. Now, we can identify every $I_x = ]a_x,b_x[$ to its pair of points $(a_x,b_x)\in\Q\times\Q$, and as $\Q$ and its powers are countable, we know that there are countably many distinct rational open intervals. So we conclude that there must exists a countable subset $\Omega'\subseteq\Omega$ such that $\bigcup_{x\in \Omega'}I_{x} = \bigcup_{x\in \Omega}I_{x}$ (what we do here is basically removing the - possibly uncountable - unnecessary repetitions). Indexing this countable subset yields a countable subfamily $\left\{I_{k}\right\}_{k=1}^{\infty}$ such that $$\bigcup_{k=1}^{\infty} I_{k}=\bigcup_{\alpha \in A} I_{\alpha} $$
		\item[(ii)]
		Define $\sim$ on $\Omega$ as $a\sim b \Leftrightarrow [a,b]\subset\Omega$ or $[b,a]\subset\Omega$. We first show that $\sim$ defines an equivalence relation:
		\begin{enumerate}
			\item{Reflexivity:}
			$\forall a\in\Omega, [a,a] = \{a\} \subset\Omega$
			\item{Symetry:}
			$\forall a,b\in\Omega, \ a\sim b \Rightarrow [a,b]\subset\Omega \text{  or  } [b,a]\subset\Omega \Rightarrow b\sim a$
			\item{Transitivity:}
			$\forall a,b,c\in\Omega$ such that $a\sim b$ and $b\sim c$, we have $\equirel ab$ or $\equirel ba$, and $\equirel bc$ or $\equirel cb$.\\			
			We assume here $a,b,c$ are distinct; if either $a=b, a=c, \text{ or } b=c$, we directly have $a\sim c$.\\
			Then we have: \begin{align*}
				\equirel ab \text{ and } \equirel bc &\Rightarrow \equirel ac \Rightarrow a\sim c\\
				\equirel ab \text{ and } \equirel cb &\Rightarrow \equirel ac \ \text{or} \ \equirel ca \Rightarrow a\sim c\\
				\equirel ba \text{ and } \equirel cb &\Rightarrow \equirel ca \Rightarrow a\sim c\\
				\equirel ba \text{ and } \equirel bc &\Rightarrow \equirel ca \ \text{or} \ \equirel ac \Rightarrow a\sim c
			\end{align*}
		\end{enumerate}
		We can now consider the equivalence class of $a\in\Omega$, here denoted $[a]$. We want to show that $[a]$ is an open interval containing $a$.

		The 'interval' part is quite clear: since $\forall b,c\in[a], \forall d\in[b,c]$ (assuming WLOG that $b\leq c$), we have by transitivity $d\in[a]$ so $[b,c]\subset[a]$, which means $[a]$ is connected, thus an interval.
		
		For the 'open' part, we can use the fact that $\Omega$ is open, so $\forall b\in[a]$, there exists a neighborhood $]b-\delta, b+\delta[\subseteq\Omega$. By transitivity again, $\forall c\in]b-\delta,b+\delta[$, we have $c\in[b]=[a]$, so $]b-\delta, b+\delta[\subseteq[a]$. As $\forall b\in[a]$ there exists a neighborhood of $b$ contained in $[a]$, we conclude $[a]$ is open.
		
		So as $\Omega=\bigcup_{x\in\Omega}[x]$ and $[x]$ are open intervals, we can use (i) to extract a countable subfamily. Two equivalence classes are either equals or disjoints, hence we found a way to write $\Omega$ as a countable union of disjoint open intervals.\\

		\item[(iii)]
		We begin by proving this useful claim:
		\begin{claim}
			$\forall x\in\R^n,\ \forall\delta>0,\ \exists a,b\in\Q^n$ such that $x\in\Box(a,b):=\prod_{i=1}^n\left]a_i, b_i\right[$ and $\Box(a,b)\subset B(x,\delta)$
		\end{claim}
		\begin{proof}
			As we want $x\in\Box(a,b)$, we need to have $a_i<x_i<b_i$, $1\leq i\leq n$. Moreover, as we want $\Box(a,b)\subset B(x,\delta)$, we need each $a_i,b_i$ to be not too far apart, say close enough to guarantee $b_i-a_i<\delta/\sqrt n$,  $1\leq i\leq n$.
			Similarly to (i), by density of $\Q$ in $\R$, we can find such $a_i,b_i\in\Q$,  $1\leq i\leq n$.

			We now have to check that our guess about the size of the box is sufficient: we want to show that $\forall x'\in\Box(a,b), x'\in B(x,\delta)$, i.e. $\left\|x-x'\right\|<\delta$.\\
			As we have $a_i<x'_i<b_i$ and $a_i<x_i<b_i$, $1\leq i\leq n$, we can get the upper bound $\left|x_i-x'_i\right| < b_i-a_i < \delta/\sqrt n$, $1\leq i\leq n$. We're done, as $$\left\|x-x'\right\|=\sqrt{\sum_{i=1}^n(x_i-x'_i)^2}<\sqrt{\sum_{i=1}^n(\delta/\sqrt n)^2} = \sqrt{n\cdot\delta^2/n}=\delta$$
		\end{proof}
		As $\Omega\subset\R^n$ is open, $\forall x\in\R^n,\ \exists\delta_x>0$ such that $B(x,\delta_x)\subset\Omega$. We can now write $\Omega=\bigcup_{x\in\Omega}B(x,\delta_x)$, which we can rewrite as $\Omega=\bigcup_{x\in\Omega}\Box(a_x,b_x)$ since for a ball $B(x,\delta_x)$, the above claim gives us $a_x,b_x\in\Q^n$ such that $x\in\Box(a_x,b_x)\subset B(x,\delta_x)$. Using now a very similar argument to (i), as $\Q^n\times\Q^n$ is countable, there exists at most countably many distinct rational open boxes, so the previous union can be made countable.
	\end{enumerate}
\end{solution}

\end{document}