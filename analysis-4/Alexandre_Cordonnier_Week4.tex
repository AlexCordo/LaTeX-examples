\documentclass[12pt]{amsart}

\usepackage{amssymb,amsmath,amsthm, amsfonts}
%\usepackage[german]{babel}
\usepackage[latin1]{inputenc}
\usepackage[all,knot]{xy}
\usepackage{graphicx}
\usepackage{tikz}

\oddsidemargin = 0.5cm 
\evensidemargin = 0.5cm 
\textwidth = 16cm
\headsep = -1cm
\textheight = 22cm

\newtheorem{case}{Case}
\newcommand{\caseif}{\textnormal{if }}
\newcommand{\leg}[2]{\genfrac{(}{)}{}{}{#1}{#2}}
\newcommand{\bfrac}[2]{\genfrac{}{}{}{0}{#1}{#2}}
\newcommand{\sm}[4]{\left(\begin{smallmatrix}#1&#2\\ #3&#4 \end{smallmatrix} \right)}
\newtheorem{theorem}{Theorem}
\newtheorem{lemma}[theorem]{Lemma}
\newtheorem{corollary}[theorem]{Corollary}
\newtheorem*{conjecture}{\bf Conjecture}
\newtheorem{proposition}[theorem]{Proposition}
\newtheorem{definition}[theorem]{Definition}
%\renewcommand{\theequation}{\thesection.\arabic{equation}}
\renewcommand{\thetheorem}{\thesection.\arabic{theorem}}
\theoremstyle{definition}
\newtheorem{exercise}{Exercise}
\newtheorem{bonus_exercise}[exercise]{Exercise*}
\newtheorem*{solution}{Solution}
\newtheorem*{answer}{Answer}
\theoremstyle{remark}
\newtheorem*{claim}{Claim}
%\newtheorem*{proof}{Proof}
\newtheorem*{theoremno}{{\bf Theorem}}
\newtheorem*{remark}{Remark}
\newtheorem*{hint}{Hint}
\newtheorem*{example}{Example}
\numberwithin{theorem}{section}
% \numberwithin{equation}{section}
\newtheorem*{theorem*}{Theorem}

\newcommand{\C}{\mathbb{C}}
\newcommand{\R}{\mathbb{R}}
\newcommand{\N}{\mathbb{N}}
\newcommand{\Q}{\mathbb{Q}}
\newcommand{\Z}{\mathbb{Z}}
\def\Box{\operatorname{Box}}
\def\Vol{\operatorname{Vol}}
\newcommand{\m}{\mathrm{m}}
\def\c{^{\mathsf{c}}}
\newcommand{\eps}{\varepsilon}
\def\a{\tilde{a}}


\begin{document}

\noindent
Name: Alexandre Cordonnier \\
SCIPER: 310692 \\
\begin{center}
Exercises - Week 4 \\
Analysis IV MATH-205 \\ Spring 2021
\end{center}

\stepcounter{exercise}
\begin{exercise}
	Let $\Omega \subseteq \mathbb{R}^{n}$ measurable and let $f: \Omega \rightarrow[0, \infty[$ be a nonnegative and integrable function. If $\alpha>0$ and $E_{\alpha}:=\{x \in \Omega: f(x)>\alpha\},$ prove that $$\mathrm{m}\left(E_{\alpha}\right) \leq \frac{1}{\alpha} \int_{\Omega} fdx$$
\end{exercise}

\begin{solution}
	We start by noticing that for all $\alpha>0$, $E_{\alpha}$ is a measurable set as $f$ is measurable and $E_{\alpha} = \{x \in \Omega: f(x)>\alpha\} = f^{-1}(]\alpha, +\infty[)$ with $]\alpha, +\infty[$ being an open subset of $[0,+\infty[$.\\
	For $\alpha>0$, we have $\forall x\in E_\alpha, \alpha < f(x)$. So by monotonicity of the integral, we have $$\int_{E_\alpha}\alpha dx \leq \int_{E_\alpha}f(x)dx$$
	Moreover, as $E_\alpha\subseteq\Omega$ and $f$ is absolutely integrable, we have $$\int_{E_\alpha}f(x)dx \leq \int_\Omega f(x)dx < \infty$$
	Evaluating $\int_{E_\alpha}\alpha dx$ yields the wanted inequality :
	\begin{align*}
		\int_\Omega f(x)dx &\geq \int_{E_\alpha}\alpha dx\\
		&= \alpha\cdot\int_{E_\alpha}dx\\
		&= \alpha\cdot\int_{\Omega} \mathbf{1}_{E_\alpha}dx\\
		&= \alpha\cdot\m(E_\alpha)
	\end{align*}
	Therefore we have $\m(E_\alpha) \leq \frac1\alpha\int_\Omega f(x)dx$ as $\alpha>0$.
\end{solution}

\stepcounter{exercise}
\begin{exercise}
	We begin by defining
	$$
	\begin{array}{c}
	P_{0}:=[0,1] \\
	P_{1}:=[0,1 / 3] \cup[2 / 3,1] \\
	P_{2}:=[0,1 / 9] \cup[2 / 9,1 / 3] \cup[2 / 3,7 / 9] \cup[8 / 9,1]
	\end{array}
	$$
	and so on (at each step we eliminate an interval of lenght $1 / 3^{k+1}$ in the middle of each interval of lenght $1 / 3^{k}$ ) such that $$\cdots \subset P_{k+1} \subset P_{k} \subset \cdots \subset P_{2} \subset P_{1} \subset P_{0}$$
	The Cantor set is defined as $$P:=\bigcap_{k=0}^{\infty} P_{k}$$
	Show the following properties:
	\begin{enumerate}
		\item[(i)]
		$P$ is compact.
		\item[(ii)]
		$P$ is measurable of Lebesgue measure 0 .
		\item[(iii)]
		$P=\left\{a \in[0,1]: a=0 . a_{1} a_{2} \cdots\text{ with } a_{i} \in\{0,2\}\right\}$, where $0.a_1a_2\cdots$ denotes \textbf{a} possible tenary
		expansion of $a \in[0,1]$ (remember that this expansion is not unique, see the previous exercise sheet).
		\item[(iv)]
		$P$ is uncountable.
	\end{enumerate}
\end{exercise}

\begin{solution}
	\begin{enumerate}
		\item[(i)]
		By defintion, $P_n$ is a union of $2^n$ disjoint closed intervals (each of lenght $3^{-n}$), so $P_n$ is closed as a finite union of closed sets. Moreover, we have that $P\c=\left(\bigcap_{k=0}^\infty P_k\right)\c = \bigcup_{k=1}^\infty P_k\c$ is open as a countable union of open sets, hence $P$ is closed.\\
		As $P_n\subseteq P_{n-1}\ \forall n\in\N$, we have $P\subset P_0 = [0,1]$, so by Heine-Borel theorem, $P$ is compact as it is closed \& bounded.
		\item[(ii)]
		$P$ is measurable as $P$ is closed. As $P_k$ is a union of $2^n$ disjoint closed intervals of lenght $3^{-n}$, we have $\m(P_k)=2^k\cdot 3^{-k} = (2/3)^k$. As $\{P_k\}_{k=0}^\infty$ is a collection of nested closed sets with $\m(P_0) = \m([0,1]) = 1$, we can use result of Exercise 5 of Sheet 3 to conclude : $$\m(P) = \m\left(\bigcap_{k=0}^\infty P_k\right) = \lim_{n\to\infty}\m(P_n) = \lim_{n\to\infty} \left(\frac23\right)^n = 0$$
		\item[(iii)]
		We prove $P=\left\{a \in[0,1]: a=0 . a_{1} a_{2} \cdots\text{ with } a_{i} \in\{0,2\}\right\}$ by double inclusion:
		\begin{itemize}
			\item[$(\subseteq)$]
			By construction, we have $P_k=\bigcup_{i=1}^{2^k}P_{k,i}$, where sets $P_{k,i}$ are disjoint closed intervals of length $3^{-k}$. We assume they are ordered in the natural way, in the sense that $\max P_{k,i} < \min P_{k,i+1}$. For example, $P_1=P_{1,1}\cup P_{1,2}$ where $P_{1,1}=[0,1/3], P_{1,2}=[2/3, 1]$ are disjoint closed intervals of length $3^{-1} = 1/3$. We can now write $$P=\bigcap_{k=0}^\infty\bigcup_{i=1}^{2^k}P_{k,i}$$
			Take any $a\in P=\bigcap_{k=0}^\infty\bigcup_{i=1}^{2^k}P_{k,i}$, then by definition $\forall k\geq 0,\ \exists i=i(k) \in\{1,2,3,\ldots,2^k\}$ such that $a\in P_{k,i(k)}$. So we now have for $a\in P$ a collection $\left\{P_{k,i(k)}\right\}_{k=0}^\infty$ such that $a\in P_{k,i(k)}\ \forall k\geq0$. 
			
			We now define the sequence $\{a_k\}_{k=1}^\infty$ :
			$$a_k = \begin{cases}
				2	& \text{if }P_{k,i(k)-1}\subset P_{k-1, i(k-1)}\\
				0	& \text{else}
			\end{cases}$$
			(we adopt the convention that $\forall k\geq0,\ P_{k,0}\nsubseteq P_0$, for instance by setting $P_{k,0}=[-2,-1]$)

			What we do here is assign $2$ to $a_k$ if $a$ is in the right third of the interval of $P_{k-1}$ containing $a$, i.e. if the next interval on the left of the interval of $P_k$ containing $a$ is contained in the interval of $P_{k-1}$ containing $a$. In reverse, if the next interval on the left of the interval of $P_k$ containing $a$ \emph{is not} contained in the interval of $P_{k-1}$ containing $a$, it means that $a$ is in the \emph{left} third of the interval of $P_{k-1}$ containing $a$, so we assign $0$ to $a_k$.

			Now, by construction of $\{a_k\}_{k=1}^\infty$, we have $\sum_{k=1}^n \frac{a_k}{3^k}\in P_{n,i(n)}$, as well as $a\in P_{n,i(n)}$ so $\left|a-\sum_{k=1}^n \frac{a_k}{3^k}\right|\leq 3^{-n}$. We can thus conclude that $a=0.a_1a_2a_3\ldots$, hence as $a_i\in\{0,2\}$ we have $a\in\left\{a \in[0,1]: a=0 . a_{1} a_{2} \cdots\text{ with } a_{i} \in\{0,2\}\right\}$
			\item[$(\supseteq)$]
			Let $a=0.a_1a_2a_3\ldots\in[0,1]$ such that $a_i\in\{0,2\}$. Similarly to Exercise 6 of Sheet 3, we note $\a_n = \sum_{k=1}^n \frac{a_k}{3^k}$ and we have $0\leq a-\a_n\leq3^{-n}$ i.e. $a\in[\a_n, \a_n+3^{-n}]$.
			It suffices to show that $[\a_n, \a_n+3^{-n}]\subseteq P_n\forall n\in\N$, because then $a\in P_n\forall n\in\N \implies a\in\bigcap_{n\in\N}=P$. We proceed by induction on $n$:
			\begin{itemize}
				\item[$n=1$ : ]
				$\a_1=0$ or $2/3$ as $a_1\in\{0,2\}$, so as $[0,1/3],[2/3,1]\subseteq [0,1/3]\cup[2/3,1]$, we have $[\a_1, \a_1+1/3]\subseteq P_1\subseteq P_0$.
				\item[$n\geq1$ : ]
				By definition, "\emph{we eliminate an interval of lenght $1/3k+1$ in the middle of each interval of lenght $1/3k$}", so we have \begin{align*}
					[\a_n,\a_n+3^{-n}]\subseteq P_n &\implies [\a_n,\a_n+3^{-n-1}]\cup[\a_n+2\cdot3^{-n-1},\a_n+3^{-n-1}]\subseteq P_{n+1}\\
					&\implies [\a_{n+1},\a_{n+1}+3^{-(n+1)}]\subseteq P_{n+1}
				\end{align*}
				as $\a_{n+1}=\a_n$ or $\a_n+2\cdot3^{-(n+1)}$, since $a_{n+1}\in\{0,2\}$.
			\end{itemize}
		\end{itemize}
		We conclude finally that $P=\left\{a \in[0,1]: a=0 . a_{1} a_{2} \cdots\text{ with } a_{i} \in\{0,2\}\right\}$.
		\item[(iv)]
		We note $T:=\left\{a \in[0,1]: a=0 . a_{1} a_{2} \cdots\text{ with } a_{i} \in\{0,2\}\right\}$.
		We basically use Cantor argument to show that $T = P$ is uncountable.\\
		Suppose $T$ is countable. Then we can index its elements such that $T=\{a^{(j)}\}_{j\in\N}$. We want to construct an element $a = 0.a_1a_2a_3\cdots$ of $T$ such that $a\notin\{a^{(j)}\}_{j\in\N}$. The idea is to pick, for $k\in\N$, $a_k \in\{0,2\}$ such that $a_k \neq a_k^{(k)}$, i.e. $a_k = 2 - a_k^{(k)}$. Indeed, if $a_k \neq a_k^{(k)}\ \forall k\in\N$, then $a$ has at least one different "decimal" to any $a^{(k)}\in\{a^{(j)}\}_{j\in\N}$, which means $a$ is different to all $a^{(k)}\in\{a^{(j)}\}_{j\in\N}$, i.e. $a\notin\{a^{(j)}\}_{j\in\N}$. We found the sought constradiction.
		Explicitely, we have $$a = \sum_{j=1}^\infty \frac{2-a_j^{(j)}}{3^j}$$
	\end{enumerate}
\end{solution}
\end{document}